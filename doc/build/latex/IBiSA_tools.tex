% Generated by Sphinx.
\def\sphinxdocclass{report}
\documentclass[letterpaper,10pt,english]{sphinxmanual}

\usepackage[utf8]{inputenc}
\ifdefined\DeclareUnicodeCharacter
  \DeclareUnicodeCharacter{00A0}{\nobreakspace}
\else\fi
\usepackage{cmap}
\usepackage[T1]{fontenc}
\usepackage{amsmath,amssymb,amstext}
\usepackage{babel}
\usepackage{times}
\usepackage[Bjarne]{fncychap}
\usepackage{longtable}
\usepackage{sphinx}
\usepackage{multirow}
\usepackage{eqparbox}


\addto\captionsenglish{\renewcommand{\figurename}{Fig.\@ }}
\addto\captionsenglish{\renewcommand{\tablename}{Table }}
\SetupFloatingEnvironment{literal-block}{name=Listing }

\addto\extrasenglish{\def\pageautorefname{page}}

\setcounter{tocdepth}{1}


\title{IBiSA\_tools Documentation}
\date{Oct 24, 2016}
\release{1.00}
\author{Kota Kasahara}
\newcommand{\sphinxlogo}{}
\renewcommand{\releasename}{Release}
\makeindex

\makeatletter
\def\PYG@reset{\let\PYG@it=\relax \let\PYG@bf=\relax%
    \let\PYG@ul=\relax \let\PYG@tc=\relax%
    \let\PYG@bc=\relax \let\PYG@ff=\relax}
\def\PYG@tok#1{\csname PYG@tok@#1\endcsname}
\def\PYG@toks#1+{\ifx\relax#1\empty\else%
    \PYG@tok{#1}\expandafter\PYG@toks\fi}
\def\PYG@do#1{\PYG@bc{\PYG@tc{\PYG@ul{%
    \PYG@it{\PYG@bf{\PYG@ff{#1}}}}}}}
\def\PYG#1#2{\PYG@reset\PYG@toks#1+\relax+\PYG@do{#2}}

\expandafter\def\csname PYG@tok@gd\endcsname{\def\PYG@tc##1{\textcolor[rgb]{0.63,0.00,0.00}{##1}}}
\expandafter\def\csname PYG@tok@gu\endcsname{\let\PYG@bf=\textbf\def\PYG@tc##1{\textcolor[rgb]{0.50,0.00,0.50}{##1}}}
\expandafter\def\csname PYG@tok@gt\endcsname{\def\PYG@tc##1{\textcolor[rgb]{0.00,0.27,0.87}{##1}}}
\expandafter\def\csname PYG@tok@gs\endcsname{\let\PYG@bf=\textbf}
\expandafter\def\csname PYG@tok@gr\endcsname{\def\PYG@tc##1{\textcolor[rgb]{1.00,0.00,0.00}{##1}}}
\expandafter\def\csname PYG@tok@cm\endcsname{\let\PYG@it=\textit\def\PYG@tc##1{\textcolor[rgb]{0.25,0.50,0.56}{##1}}}
\expandafter\def\csname PYG@tok@vg\endcsname{\def\PYG@tc##1{\textcolor[rgb]{0.73,0.38,0.84}{##1}}}
\expandafter\def\csname PYG@tok@vi\endcsname{\def\PYG@tc##1{\textcolor[rgb]{0.73,0.38,0.84}{##1}}}
\expandafter\def\csname PYG@tok@mh\endcsname{\def\PYG@tc##1{\textcolor[rgb]{0.13,0.50,0.31}{##1}}}
\expandafter\def\csname PYG@tok@cs\endcsname{\def\PYG@tc##1{\textcolor[rgb]{0.25,0.50,0.56}{##1}}\def\PYG@bc##1{\setlength{\fboxsep}{0pt}\colorbox[rgb]{1.00,0.94,0.94}{\strut ##1}}}
\expandafter\def\csname PYG@tok@ge\endcsname{\let\PYG@it=\textit}
\expandafter\def\csname PYG@tok@vc\endcsname{\def\PYG@tc##1{\textcolor[rgb]{0.73,0.38,0.84}{##1}}}
\expandafter\def\csname PYG@tok@il\endcsname{\def\PYG@tc##1{\textcolor[rgb]{0.13,0.50,0.31}{##1}}}
\expandafter\def\csname PYG@tok@go\endcsname{\def\PYG@tc##1{\textcolor[rgb]{0.20,0.20,0.20}{##1}}}
\expandafter\def\csname PYG@tok@cp\endcsname{\def\PYG@tc##1{\textcolor[rgb]{0.00,0.44,0.13}{##1}}}
\expandafter\def\csname PYG@tok@gi\endcsname{\def\PYG@tc##1{\textcolor[rgb]{0.00,0.63,0.00}{##1}}}
\expandafter\def\csname PYG@tok@gh\endcsname{\let\PYG@bf=\textbf\def\PYG@tc##1{\textcolor[rgb]{0.00,0.00,0.50}{##1}}}
\expandafter\def\csname PYG@tok@ni\endcsname{\let\PYG@bf=\textbf\def\PYG@tc##1{\textcolor[rgb]{0.84,0.33,0.22}{##1}}}
\expandafter\def\csname PYG@tok@nl\endcsname{\let\PYG@bf=\textbf\def\PYG@tc##1{\textcolor[rgb]{0.00,0.13,0.44}{##1}}}
\expandafter\def\csname PYG@tok@nn\endcsname{\let\PYG@bf=\textbf\def\PYG@tc##1{\textcolor[rgb]{0.05,0.52,0.71}{##1}}}
\expandafter\def\csname PYG@tok@no\endcsname{\def\PYG@tc##1{\textcolor[rgb]{0.38,0.68,0.84}{##1}}}
\expandafter\def\csname PYG@tok@na\endcsname{\def\PYG@tc##1{\textcolor[rgb]{0.25,0.44,0.63}{##1}}}
\expandafter\def\csname PYG@tok@nb\endcsname{\def\PYG@tc##1{\textcolor[rgb]{0.00,0.44,0.13}{##1}}}
\expandafter\def\csname PYG@tok@nc\endcsname{\let\PYG@bf=\textbf\def\PYG@tc##1{\textcolor[rgb]{0.05,0.52,0.71}{##1}}}
\expandafter\def\csname PYG@tok@nd\endcsname{\let\PYG@bf=\textbf\def\PYG@tc##1{\textcolor[rgb]{0.33,0.33,0.33}{##1}}}
\expandafter\def\csname PYG@tok@ne\endcsname{\def\PYG@tc##1{\textcolor[rgb]{0.00,0.44,0.13}{##1}}}
\expandafter\def\csname PYG@tok@nf\endcsname{\def\PYG@tc##1{\textcolor[rgb]{0.02,0.16,0.49}{##1}}}
\expandafter\def\csname PYG@tok@si\endcsname{\let\PYG@it=\textit\def\PYG@tc##1{\textcolor[rgb]{0.44,0.63,0.82}{##1}}}
\expandafter\def\csname PYG@tok@s2\endcsname{\def\PYG@tc##1{\textcolor[rgb]{0.25,0.44,0.63}{##1}}}
\expandafter\def\csname PYG@tok@nt\endcsname{\let\PYG@bf=\textbf\def\PYG@tc##1{\textcolor[rgb]{0.02,0.16,0.45}{##1}}}
\expandafter\def\csname PYG@tok@nv\endcsname{\def\PYG@tc##1{\textcolor[rgb]{0.73,0.38,0.84}{##1}}}
\expandafter\def\csname PYG@tok@s1\endcsname{\def\PYG@tc##1{\textcolor[rgb]{0.25,0.44,0.63}{##1}}}
\expandafter\def\csname PYG@tok@ch\endcsname{\let\PYG@it=\textit\def\PYG@tc##1{\textcolor[rgb]{0.25,0.50,0.56}{##1}}}
\expandafter\def\csname PYG@tok@m\endcsname{\def\PYG@tc##1{\textcolor[rgb]{0.13,0.50,0.31}{##1}}}
\expandafter\def\csname PYG@tok@gp\endcsname{\let\PYG@bf=\textbf\def\PYG@tc##1{\textcolor[rgb]{0.78,0.36,0.04}{##1}}}
\expandafter\def\csname PYG@tok@sh\endcsname{\def\PYG@tc##1{\textcolor[rgb]{0.25,0.44,0.63}{##1}}}
\expandafter\def\csname PYG@tok@ow\endcsname{\let\PYG@bf=\textbf\def\PYG@tc##1{\textcolor[rgb]{0.00,0.44,0.13}{##1}}}
\expandafter\def\csname PYG@tok@sx\endcsname{\def\PYG@tc##1{\textcolor[rgb]{0.78,0.36,0.04}{##1}}}
\expandafter\def\csname PYG@tok@bp\endcsname{\def\PYG@tc##1{\textcolor[rgb]{0.00,0.44,0.13}{##1}}}
\expandafter\def\csname PYG@tok@c1\endcsname{\let\PYG@it=\textit\def\PYG@tc##1{\textcolor[rgb]{0.25,0.50,0.56}{##1}}}
\expandafter\def\csname PYG@tok@o\endcsname{\def\PYG@tc##1{\textcolor[rgb]{0.40,0.40,0.40}{##1}}}
\expandafter\def\csname PYG@tok@kc\endcsname{\let\PYG@bf=\textbf\def\PYG@tc##1{\textcolor[rgb]{0.00,0.44,0.13}{##1}}}
\expandafter\def\csname PYG@tok@c\endcsname{\let\PYG@it=\textit\def\PYG@tc##1{\textcolor[rgb]{0.25,0.50,0.56}{##1}}}
\expandafter\def\csname PYG@tok@mf\endcsname{\def\PYG@tc##1{\textcolor[rgb]{0.13,0.50,0.31}{##1}}}
\expandafter\def\csname PYG@tok@err\endcsname{\def\PYG@bc##1{\setlength{\fboxsep}{0pt}\fcolorbox[rgb]{1.00,0.00,0.00}{1,1,1}{\strut ##1}}}
\expandafter\def\csname PYG@tok@mb\endcsname{\def\PYG@tc##1{\textcolor[rgb]{0.13,0.50,0.31}{##1}}}
\expandafter\def\csname PYG@tok@ss\endcsname{\def\PYG@tc##1{\textcolor[rgb]{0.32,0.47,0.09}{##1}}}
\expandafter\def\csname PYG@tok@sr\endcsname{\def\PYG@tc##1{\textcolor[rgb]{0.14,0.33,0.53}{##1}}}
\expandafter\def\csname PYG@tok@mo\endcsname{\def\PYG@tc##1{\textcolor[rgb]{0.13,0.50,0.31}{##1}}}
\expandafter\def\csname PYG@tok@kd\endcsname{\let\PYG@bf=\textbf\def\PYG@tc##1{\textcolor[rgb]{0.00,0.44,0.13}{##1}}}
\expandafter\def\csname PYG@tok@mi\endcsname{\def\PYG@tc##1{\textcolor[rgb]{0.13,0.50,0.31}{##1}}}
\expandafter\def\csname PYG@tok@kn\endcsname{\let\PYG@bf=\textbf\def\PYG@tc##1{\textcolor[rgb]{0.00,0.44,0.13}{##1}}}
\expandafter\def\csname PYG@tok@cpf\endcsname{\let\PYG@it=\textit\def\PYG@tc##1{\textcolor[rgb]{0.25,0.50,0.56}{##1}}}
\expandafter\def\csname PYG@tok@kr\endcsname{\let\PYG@bf=\textbf\def\PYG@tc##1{\textcolor[rgb]{0.00,0.44,0.13}{##1}}}
\expandafter\def\csname PYG@tok@s\endcsname{\def\PYG@tc##1{\textcolor[rgb]{0.25,0.44,0.63}{##1}}}
\expandafter\def\csname PYG@tok@kp\endcsname{\def\PYG@tc##1{\textcolor[rgb]{0.00,0.44,0.13}{##1}}}
\expandafter\def\csname PYG@tok@w\endcsname{\def\PYG@tc##1{\textcolor[rgb]{0.73,0.73,0.73}{##1}}}
\expandafter\def\csname PYG@tok@kt\endcsname{\def\PYG@tc##1{\textcolor[rgb]{0.56,0.13,0.00}{##1}}}
\expandafter\def\csname PYG@tok@sc\endcsname{\def\PYG@tc##1{\textcolor[rgb]{0.25,0.44,0.63}{##1}}}
\expandafter\def\csname PYG@tok@sb\endcsname{\def\PYG@tc##1{\textcolor[rgb]{0.25,0.44,0.63}{##1}}}
\expandafter\def\csname PYG@tok@k\endcsname{\let\PYG@bf=\textbf\def\PYG@tc##1{\textcolor[rgb]{0.00,0.44,0.13}{##1}}}
\expandafter\def\csname PYG@tok@se\endcsname{\let\PYG@bf=\textbf\def\PYG@tc##1{\textcolor[rgb]{0.25,0.44,0.63}{##1}}}
\expandafter\def\csname PYG@tok@sd\endcsname{\let\PYG@it=\textit\def\PYG@tc##1{\textcolor[rgb]{0.25,0.44,0.63}{##1}}}

\def\PYGZbs{\char`\\}
\def\PYGZus{\char`\_}
\def\PYGZob{\char`\{}
\def\PYGZcb{\char`\}}
\def\PYGZca{\char`\^}
\def\PYGZam{\char`\&}
\def\PYGZlt{\char`\<}
\def\PYGZgt{\char`\>}
\def\PYGZsh{\char`\#}
\def\PYGZpc{\char`\%}
\def\PYGZdl{\char`\$}
\def\PYGZhy{\char`\-}
\def\PYGZsq{\char`\'}
\def\PYGZdq{\char`\"}
\def\PYGZti{\char`\~}
% for compatibility with earlier versions
\def\PYGZat{@}
\def\PYGZlb{[}
\def\PYGZrb{]}
\makeatother

\renewcommand\PYGZsq{\textquotesingle}

\begin{document}

\maketitle
\tableofcontents
\phantomsection\label{index::doc}


Contents:


\chapter{About}
\label{about:about}\label{about::doc}\label{about:welcome-to-ibisa-tools-s-documentation}

\section{Authors}
\label{about:authors}
IBiSA\_tools version 1.00 (12 Jul. 2016)
\begin{itemize}
\item {} 
Kota Kasahara, Ritsumeikan University, Japan

\item {} 
Kengo Kinoshita, Tohoku University, Japan

\end{itemize}


\section{Citation}
\label{about:citation}\begin{itemize}
\item {} 
Kasahara K and Kinoshita K, (Under review) IBiSA\_tools: A Computational Toolkit for the Ion Binding State Analysis on Molecular Dynamics Trajectories of Ion Channels.

\item {} 
Kasahara K, Shirota M, and Kinoshita K (2016) Ion Concentration- and Voltage-Dependent Push and Pull Mechanisms of Potassium Channel Ion Conduction. PLoS ONE, 11, e0150716.

\item {} 
Kasahara K, Shirota M, and Kinoshita K (2013) Ion Concentration-Dependent Ion Conduction Mechanism of a Voltage-Sensitive Potassium Channel. PLoS ONE, 8, e56342.

\end{itemize}


\section{Requirements}
\label{about:requirements}\begin{itemize}
\item {} 
The current version of IBiSA\_tools can be applied only for GROMACS trajectory file (.trr). If your trajectories are written in another format, you have to convert it into .trr, by using some tools, e.g., VMD plugin and MDAnalysis (see Appendix).

\item {} 
IBiSA\_tools is consisting of a C++ program and Python (2.6 or 2.7) scripts.

\item {} 
The attached tutorial files use R software (www.r-project.org) to draw plots.

\item {} 
Network drawing software, e.g., Cytoscape, is required to visualize ion-binding state graph.

\end{itemize}


\section{Download}
\label{about:download}
\url{https://github.com/kotakasahara/IBiSA\_tools}


\section{Installation}
\label{about:installation}
Only a C++ program, trachan, must be compiled as follows:

\begin{Verbatim}[commandchars=\\\{\}]
\PYG{n}{cd} \PYG{n}{src}\PYG{o}{/}\PYG{n}{trachan}
\PYG{o}{.}\PYG{o}{/}\PYG{n}{configure}
\PYG{n}{make}
\end{Verbatim}

In this document, we assume the binary and python scripts are included in the directory indicated as \$\{IBISA\}.


\chapter{Overview}
\label{overview:overview}\label{overview::doc}
\emph{IBiSA\_tools}, which stands for ``Ion-Binding State Analysis tools'', provides a computational tools for analyzing ion conduction mechanisms hidden in the molecular dynamics (MD) trajectory data. This analysis provides a varitery of information about each ion conduction event and overall properties. See the citations for details of theory and applications.

\includegraphics[width=16cm]{{fig_overview_detail}.png}

A list of programs contained in \emph{IBiSA\_tools} (the left column in A) and the default names of input/output files (the right column in A) are summarized. The panels B, C, D, and E are schematic images of output figures (reprinted from our paper).


\chapter{Tutorial}
\label{tutorial::doc}\label{tutorial:tutorial}

\section{Setting the path to IBiSA\_tools}
\label{tutorial:setting-the-path-to-ibisa-tools}
\begin{Verbatim}[commandchars=\\\{\}]
\PYG{n}{export} \PYG{n}{IBISA}\PYG{o}{=}\PYG{l+s+s2}{\PYGZdq{}}\PYG{l+s+s2}{\PYGZdl{}}\PYG{l+s+si}{\PYGZob{}HOME\PYGZcb{}}\PYG{l+s+s2}{/local/IBiSA\PYGZus{}tools}\PYG{l+s+s2}{\PYGZdq{}}
\end{Verbatim}

Set your own install directory.


\section{Preparing the configuration file}
\label{tutorial:preparing-the-configuration-file}
config.txt:

\begin{Verbatim}[commandchars=\\\{\}]
\PYG{o}{\PYGZhy{}}\PYG{o}{\PYGZhy{}}\PYG{n}{fn}\PYG{o}{\PYGZhy{}}\PYG{n}{pdb}                      \PYG{n}{init}\PYG{o}{.}\PYG{n}{pdb}
\PYG{o}{\PYGZhy{}}\PYG{o}{\PYGZhy{}}\PYG{n}{dt}                          \PYG{l+m+mi}{10}
\PYG{o}{\PYGZhy{}}\PYG{o}{\PYGZhy{}}\PYG{n}{site}\PYG{o}{\PYGZhy{}}\PYG{n}{boundary}               \PYG{l+m+mf}{20.0}
\PYG{o}{\PYGZhy{}}\PYG{o}{\PYGZhy{}}\PYG{n}{site}\PYG{o}{\PYGZhy{}}\PYG{n}{boundary}              \PYG{o}{\PYGZhy{}}\PYG{l+m+mf}{25.0}
\PYG{o}{\PYGZhy{}}\PYG{o}{\PYGZhy{}}\PYG{n}{fn}\PYG{o}{\PYGZhy{}}\PYG{n}{pore}\PYG{o}{\PYGZhy{}}\PYG{n}{axis}\PYG{o}{\PYGZhy{}}\PYG{n}{coordinates}    \PYG{n}{pore\PYGZus{}axis}\PYG{o}{.}\PYG{n}{txt}
\PYG{o}{\PYGZhy{}}\PYG{o}{\PYGZhy{}}\PYG{n}{fn}\PYG{o}{\PYGZhy{}}\PYG{n}{pore}\PYG{o}{\PYGZhy{}}\PYG{n}{axis}\PYG{o}{\PYGZhy{}}\PYG{n}{coordinates}\PYG{o}{\PYGZhy{}}\PYG{n}{r}  \PYG{n}{pore\PYGZus{}axis\PYGZus{}r}\PYG{o}{.}\PYG{n}{txt}
\PYG{o}{\PYGZhy{}}\PYG{o}{\PYGZhy{}}\PYG{n}{pore}\PYG{o}{\PYGZhy{}}\PYG{n}{axis}\PYG{o}{\PYGZhy{}}\PYG{n}{basis}\PYG{o}{\PYGZhy{}}\PYG{k+kn}{from}        \PYG{n+nn}{A} \PYG{l+m+mi}{374} \PYG{n}{O}
\PYG{o}{\PYGZhy{}}\PYG{o}{\PYGZhy{}}\PYG{n}{pore}\PYG{o}{\PYGZhy{}}\PYG{n}{axis}\PYG{o}{\PYGZhy{}}\PYG{n}{basis}\PYG{o}{\PYGZhy{}}\PYG{k+kn}{from}        \PYG{n+nn}{B} \PYG{l+m+mi}{374} \PYG{n}{O}
\PYG{o}{\PYGZhy{}}\PYG{o}{\PYGZhy{}}\PYG{n}{pore}\PYG{o}{\PYGZhy{}}\PYG{n}{axis}\PYG{o}{\PYGZhy{}}\PYG{n}{basis}\PYG{o}{\PYGZhy{}}\PYG{k+kn}{from}        \PYG{n+nn}{C} \PYG{l+m+mi}{374} \PYG{n}{O}
\PYG{o}{\PYGZhy{}}\PYG{o}{\PYGZhy{}}\PYG{n}{pore}\PYG{o}{\PYGZhy{}}\PYG{n}{axis}\PYG{o}{\PYGZhy{}}\PYG{n}{basis}\PYG{o}{\PYGZhy{}}\PYG{k+kn}{from}        \PYG{n+nn}{D} \PYG{l+m+mi}{374} \PYG{n}{O}
\PYG{o}{\PYGZhy{}}\PYG{o}{\PYGZhy{}}\PYG{n}{pore}\PYG{o}{\PYGZhy{}}\PYG{n}{axis}\PYG{o}{\PYGZhy{}}\PYG{n}{basis}\PYG{o}{\PYGZhy{}}\PYG{n}{to}          \PYG{n}{A} \PYG{l+m+mi}{377} \PYG{n}{O}
\PYG{o}{\PYGZhy{}}\PYG{o}{\PYGZhy{}}\PYG{n}{pore}\PYG{o}{\PYGZhy{}}\PYG{n}{axis}\PYG{o}{\PYGZhy{}}\PYG{n}{basis}\PYG{o}{\PYGZhy{}}\PYG{n}{to}          \PYG{n}{B} \PYG{l+m+mi}{377} \PYG{n}{O}
\PYG{o}{\PYGZhy{}}\PYG{o}{\PYGZhy{}}\PYG{n}{pore}\PYG{o}{\PYGZhy{}}\PYG{n}{axis}\PYG{o}{\PYGZhy{}}\PYG{n}{basis}\PYG{o}{\PYGZhy{}}\PYG{n}{to}          \PYG{n}{C} \PYG{l+m+mi}{377} \PYG{n}{O}
\PYG{o}{\PYGZhy{}}\PYG{o}{\PYGZhy{}}\PYG{n}{pore}\PYG{o}{\PYGZhy{}}\PYG{n}{axis}\PYG{o}{\PYGZhy{}}\PYG{n}{basis}\PYG{o}{\PYGZhy{}}\PYG{n}{to}          \PYG{n}{D} \PYG{l+m+mi}{377} \PYG{n}{O}
\PYG{o}{\PYGZhy{}}\PYG{o}{\PYGZhy{}}\PYG{n}{site}\PYG{o}{\PYGZhy{}}\PYG{n+nb}{max}\PYG{o}{\PYGZhy{}}\PYG{n}{radius}           \PYG{l+m+mf}{10.0}
\PYG{o}{\PYGZhy{}}\PYG{o}{\PYGZhy{}}\PYG{n}{site}\PYG{o}{\PYGZhy{}}\PYG{n}{height}\PYG{o}{\PYGZhy{}}\PYG{n}{margin}         \PYG{l+m+mf}{5.0}
\PYG{o}{\PYGZhy{}}\PYG{o}{\PYGZhy{}}\PYG{n}{channel}\PYG{o}{\PYGZhy{}}\PYG{n}{chain}\PYG{o}{\PYGZhy{}}\PYG{n+nb}{id}          \PYG{n}{ABCD}
\PYG{o}{\PYGZhy{}}\PYG{o}{\PYGZhy{}}\PYG{n}{trace}\PYG{o}{\PYGZhy{}}\PYG{n}{atom}\PYG{o}{\PYGZhy{}}\PYG{n}{name}           \PYG{n}{K}
\PYG{o}{\PYGZhy{}}\PYG{o}{\PYGZhy{}}\PYG{n}{fn}\PYG{o}{\PYGZhy{}}\PYG{n}{trr} \PYG{n}{traj}\PYG{o}{.}\PYG{n}{trr}
\end{Verbatim}

Each line indicate a set of key and values.
\begin{itemize}
\item {} 
\emph{--fn-pdb} is the initial structure file.

\item {} 
\emph{--dt} is the time step of the trajectory file

\item {} 
\emph{--site-boundary} is specified two values, 20.0 and -25.0. This setting means that the range from 20.0 to -25.0 in the pore axis will be analyzed. The origin of the pore axis is set by \emph{--pore-axis-basis-from} key.

\item {} 
\emph{--fn-pore-axis-coordinates} and \emph{--fn-pore-axis-coordinates-r} are the output file names.

\item {} 
\emph{--pore-axis-basis-from} specifies the origin of the pore axis. ``A 374 O'' means the O atom in the residue 374 of the chain A. In this tutorial, four oxigen atoms are specified. The center of these atoms is set to be the origin of the pore axis.

\item {} 
\emph{--pore-axis-basis-to} defines the direction of the pore axis. The line from the center of \emph{...-from} to the center of \emph{...-to} defines the pore axis.

\item {} 
\emph{--channel-chain-id} specifies chain IDs of a channel protein.

\item {} 
\emph{--trace-atom-name} specifies the atom name of target ions, defined in the .pdb file.

\item {} 
\emph{--fn-trr} is the file name of the trajectory. Multiple files can be specified.

\item {} 
\emph{--site-max-radius (optional)} is the maximum radius of ion channel pore. As the channel pore is narrowed by the protein structure and the range of ion-binding sites on the radial coordinates is sterically defined by the protein structure in usual cases, this setting is not important, but is should be set large value enough to cover the pore.

\item {} 
\emph{--site-height-margin (optional)} is the margin length along the pore axis. In this case, the range from 20.0+5.0 to -25.0-5.0 will be analyzed.

\end{itemize}


\section{Converting a trajectory into the pore axis coordinates}
\label{tutorial:converting-a-trajectory-into-the-pore-axis-coordinates}
\begin{Verbatim}[commandchars=\\\{\}]
\PYGZdl{}IBISA/bin/trachan \PYGZhy{}\PYGZhy{}fn\PYGZhy{}cfg config.txt
\end{Verbatim}

Some text should appear in the standard output:

\begin{Verbatim}[commandchars=\\\{\}]
\PYG{o}{\PYGZhy{}}\PYG{o}{\PYGZhy{}}\PYG{o}{\PYGZhy{}}\PYG{o}{\PYGZhy{}}\PYG{o}{\PYGZhy{}}\PYG{o}{\PYGZhy{}}\PYG{o}{\PYGZhy{}}\PYG{o}{\PYGZhy{}}\PYG{o}{\PYGZhy{}}\PYG{o}{\PYGZhy{}}\PYG{o}{\PYGZhy{}}\PYG{o}{\PYGZhy{}}\PYG{o}{\PYGZhy{}}\PYG{o}{\PYGZhy{}}\PYG{o}{\PYGZhy{}}\PYG{o}{\PYGZhy{}}\PYG{o}{\PYGZhy{}}\PYG{o}{\PYGZhy{}}\PYG{o}{\PYGZhy{}}\PYG{o}{\PYGZhy{}}\PYG{o}{\PYGZhy{}}\PYG{o}{\PYGZhy{}}\PYG{o}{\PYGZhy{}}\PYG{o}{\PYGZhy{}}\PYG{o}{\PYGZhy{}}\PYG{o}{\PYGZhy{}}\PYG{o}{\PYGZhy{}}\PYG{o}{\PYGZhy{}}\PYG{o}{\PYGZhy{}}\PYG{o}{\PYGZhy{}}\PYG{o}{\PYGZhy{}}\PYG{o}{\PYGZhy{}}\PYG{o}{\PYGZhy{}}\PYG{o}{\PYGZhy{}}\PYG{o}{\PYGZhy{}}\PYG{o}{\PYGZhy{}}\PYG{o}{\PYGZhy{}}\PYG{o}{\PYGZhy{}}\PYG{o}{\PYGZhy{}}\PYG{o}{\PYGZhy{}}\PYG{o}{\PYGZhy{}}\PYG{o}{\PYGZhy{}}\PYG{o}{\PYGZhy{}}\PYG{o}{\PYGZhy{}}\PYG{o}{\PYGZhy{}}\PYG{o}{\PYGZhy{}}\PYG{o}{\PYGZhy{}}\PYG{o}{\PYGZhy{}}\PYG{o}{\PYGZhy{}}\PYG{o}{\PYGZhy{}}\PYG{o}{\PYGZhy{}}\PYG{o}{\PYGZhy{}}\PYG{o}{\PYGZhy{}}\PYG{o}{\PYGZhy{}}\PYG{o}{\PYGZhy{}}\PYG{o}{\PYGZhy{}}\PYG{o}{\PYGZhy{}}\PYG{o}{\PYGZhy{}}\PYG{o}{\PYGZhy{}}\PYG{o}{\PYGZhy{}}\PYG{o}{\PYGZhy{}}\PYG{o}{\PYGZhy{}}\PYG{o}{\PYGZhy{}}\PYG{o}{\PYGZhy{}}\PYG{o}{\PYGZhy{}}
                      \PYG{n}{TraChan}
  \PYG{n}{TRAjectory} \PYG{n}{analyzer} \PYG{k}{for} \PYG{n}{CHANnel} \PYG{n}{pore} \PYG{n}{axis}
\PYG{o}{\PYGZhy{}}\PYG{o}{\PYGZhy{}}\PYG{o}{\PYGZhy{}}\PYG{o}{\PYGZhy{}}\PYG{o}{\PYGZhy{}}\PYG{o}{\PYGZhy{}}\PYG{o}{\PYGZhy{}}\PYG{o}{\PYGZhy{}}\PYG{o}{\PYGZhy{}}\PYG{o}{\PYGZhy{}}\PYG{o}{\PYGZhy{}}\PYG{o}{\PYGZhy{}}\PYG{o}{\PYGZhy{}}\PYG{o}{\PYGZhy{}}\PYG{o}{\PYGZhy{}}\PYG{o}{\PYGZhy{}}\PYG{o}{\PYGZhy{}}\PYG{o}{\PYGZhy{}}\PYG{o}{\PYGZhy{}}\PYG{o}{\PYGZhy{}}\PYG{o}{\PYGZhy{}}\PYG{o}{\PYGZhy{}}\PYG{o}{\PYGZhy{}}\PYG{o}{\PYGZhy{}}\PYG{o}{\PYGZhy{}}\PYG{o}{\PYGZhy{}}\PYG{o}{\PYGZhy{}}\PYG{o}{\PYGZhy{}}\PYG{o}{\PYGZhy{}}\PYG{o}{\PYGZhy{}}\PYG{o}{\PYGZhy{}}\PYG{o}{\PYGZhy{}}\PYG{o}{\PYGZhy{}}\PYG{o}{\PYGZhy{}}\PYG{o}{\PYGZhy{}}\PYG{o}{\PYGZhy{}}\PYG{o}{\PYGZhy{}}\PYG{o}{\PYGZhy{}}\PYG{o}{\PYGZhy{}}\PYG{o}{\PYGZhy{}}\PYG{o}{\PYGZhy{}}\PYG{o}{\PYGZhy{}}\PYG{o}{\PYGZhy{}}\PYG{o}{\PYGZhy{}}\PYG{o}{\PYGZhy{}}\PYG{o}{\PYGZhy{}}\PYG{o}{\PYGZhy{}}\PYG{o}{\PYGZhy{}}\PYG{o}{\PYGZhy{}}\PYG{o}{\PYGZhy{}}\PYG{o}{\PYGZhy{}}\PYG{o}{\PYGZhy{}}\PYG{o}{\PYGZhy{}}\PYG{o}{\PYGZhy{}}\PYG{o}{\PYGZhy{}}\PYG{o}{\PYGZhy{}}\PYG{o}{\PYGZhy{}}\PYG{o}{\PYGZhy{}}\PYG{o}{\PYGZhy{}}\PYG{o}{\PYGZhy{}}\PYG{o}{\PYGZhy{}}\PYG{o}{\PYGZhy{}}\PYG{o}{\PYGZhy{}}\PYG{o}{\PYGZhy{}}\PYG{o}{\PYGZhy{}}
\PYG{n}{Copyright} \PYG{p}{(}\PYG{n}{c}\PYG{p}{)} \PYG{l+m+mi}{2016} \PYG{n}{Kota} \PYG{n}{Kasahara}\PYG{p}{,} \PYG{n}{Ritsumeikan} \PYG{n}{University}
\PYG{n}{This} \PYG{n}{software} \PYG{o+ow}{is} \PYG{n}{distributed} \PYG{n}{under} \PYG{n}{the} \PYG{n}{terms} \PYG{n}{of} \PYG{n}{the} \PYG{n}{GPL} \PYG{n}{license}

\PYG{o}{\PYGZhy{}}\PYG{o}{\PYGZhy{}}\PYG{o}{\PYGZhy{}}\PYG{o}{\PYGZhy{}}\PYG{o}{\PYGZhy{}}\PYG{o}{\PYGZhy{}}\PYG{o}{\PYGZhy{}}\PYG{o}{\PYGZhy{}}\PYG{o}{\PYGZhy{}}\PYG{o}{\PYGZhy{}}\PYG{o}{\PYGZhy{}}\PYG{o}{\PYGZhy{}}\PYG{o}{\PYGZhy{}}\PYG{o}{\PYGZhy{}}\PYG{o}{\PYGZhy{}}\PYG{o}{\PYGZhy{}}\PYG{o}{\PYGZhy{}}\PYG{o}{\PYGZhy{}}\PYG{o}{\PYGZhy{}}\PYG{o}{\PYGZhy{}}\PYG{o}{\PYGZhy{}}\PYG{o}{\PYGZhy{}}\PYG{o}{\PYGZhy{}}\PYG{o}{\PYGZhy{}}\PYG{o}{\PYGZhy{}}\PYG{o}{\PYGZhy{}}\PYG{o}{\PYGZhy{}}\PYG{o}{\PYGZhy{}}\PYG{o}{\PYGZhy{}}\PYG{o}{\PYGZhy{}}\PYG{o}{\PYGZhy{}}\PYG{o}{\PYGZhy{}}\PYG{o}{\PYGZhy{}}\PYG{o}{\PYGZhy{}}\PYG{o}{\PYGZhy{}}\PYG{o}{\PYGZhy{}}\PYG{o}{\PYGZhy{}}\PYG{o}{\PYGZhy{}}\PYG{o}{\PYGZhy{}}\PYG{o}{\PYGZhy{}}\PYG{o}{\PYGZhy{}}\PYG{o}{\PYGZhy{}}\PYG{o}{\PYGZhy{}}\PYG{o}{\PYGZhy{}}\PYG{o}{\PYGZhy{}}\PYG{o}{\PYGZhy{}}\PYG{o}{\PYGZhy{}}\PYG{o}{\PYGZhy{}}\PYG{o}{\PYGZhy{}}\PYG{o}{\PYGZhy{}}\PYG{o}{\PYGZhy{}}\PYG{o}{\PYGZhy{}}\PYG{o}{\PYGZhy{}}\PYG{o}{\PYGZhy{}}\PYG{o}{\PYGZhy{}}\PYG{o}{\PYGZhy{}}\PYG{o}{\PYGZhy{}}\PYG{o}{\PYGZhy{}}\PYG{o}{\PYGZhy{}}\PYG{o}{\PYGZhy{}}\PYG{o}{\PYGZhy{}}\PYG{o}{\PYGZhy{}}\PYG{o}{\PYGZhy{}}\PYG{o}{\PYGZhy{}}\PYG{o}{\PYGZhy{}}
\PYG{n}{This} \PYG{n}{program} \PYG{n}{contains} \PYG{n}{some} \PYG{n}{parts} \PYG{n}{of} \PYG{n}{the} \PYG{n}{source} \PYG{n}{code} \PYG{n}{of} \PYG{n}{GROMACS}
\PYG{n}{software}\PYG{o}{.} \PYG{n}{Check} \PYG{n}{out} \PYG{n}{http}\PYG{p}{:}\PYG{o}{/}\PYG{o}{/}\PYG{n}{www}\PYG{o}{.}\PYG{n}{gromacs}\PYG{o}{.}\PYG{n}{org} \PYG{n}{about} \PYG{n}{GROMACS}\PYG{o}{.}
\PYG{n}{Copyright} \PYG{p}{(}\PYG{n}{c}\PYG{p}{)} \PYG{l+m+mi}{1991}\PYG{o}{\PYGZhy{}}\PYG{l+m+mi}{2000}\PYG{p}{,} \PYG{n}{University} \PYG{n}{of} \PYG{n}{Groningen}\PYG{p}{,} \PYG{n}{The} \PYG{n}{Netherlands}
\PYG{n}{Copyright} \PYG{p}{(}\PYG{n}{c}\PYG{p}{)} \PYG{l+m+mi}{2001}\PYG{o}{\PYGZhy{}}\PYG{l+m+mi}{2010}\PYG{p}{,} \PYG{n}{The} \PYG{n}{GROMACS} \PYG{n}{development} \PYG{n}{team} \PYG{n}{at}
\PYG{n}{Uppsala} \PYG{n}{University} \PYG{o}{\PYGZam{}} \PYG{n}{The} \PYG{n}{Royal} \PYG{n}{Institute} \PYG{n}{of} \PYG{n}{Technology}\PYG{p}{,} \PYG{n}{Sweden}\PYG{o}{.}
\PYG{o}{\PYGZhy{}}\PYG{o}{\PYGZhy{}}\PYG{o}{\PYGZhy{}}\PYG{o}{\PYGZhy{}}\PYG{o}{\PYGZhy{}}\PYG{o}{\PYGZhy{}}\PYG{o}{\PYGZhy{}}\PYG{o}{\PYGZhy{}}\PYG{o}{\PYGZhy{}}\PYG{o}{\PYGZhy{}}\PYG{o}{\PYGZhy{}}\PYG{o}{\PYGZhy{}}\PYG{o}{\PYGZhy{}}\PYG{o}{\PYGZhy{}}\PYG{o}{\PYGZhy{}}\PYG{o}{\PYGZhy{}}\PYG{o}{\PYGZhy{}}\PYG{o}{\PYGZhy{}}\PYG{o}{\PYGZhy{}}\PYG{o}{\PYGZhy{}}\PYG{o}{\PYGZhy{}}\PYG{o}{\PYGZhy{}}\PYG{o}{\PYGZhy{}}\PYG{o}{\PYGZhy{}}\PYG{o}{\PYGZhy{}}\PYG{o}{\PYGZhy{}}\PYG{o}{\PYGZhy{}}\PYG{o}{\PYGZhy{}}\PYG{o}{\PYGZhy{}}\PYG{o}{\PYGZhy{}}\PYG{o}{\PYGZhy{}}\PYG{o}{\PYGZhy{}}\PYG{o}{\PYGZhy{}}\PYG{o}{\PYGZhy{}}\PYG{o}{\PYGZhy{}}\PYG{o}{\PYGZhy{}}\PYG{o}{\PYGZhy{}}\PYG{o}{\PYGZhy{}}\PYG{o}{\PYGZhy{}}\PYG{o}{\PYGZhy{}}\PYG{o}{\PYGZhy{}}\PYG{o}{\PYGZhy{}}\PYG{o}{\PYGZhy{}}\PYG{o}{\PYGZhy{}}\PYG{o}{\PYGZhy{}}\PYG{o}{\PYGZhy{}}\PYG{o}{\PYGZhy{}}\PYG{o}{\PYGZhy{}}\PYG{o}{\PYGZhy{}}\PYG{o}{\PYGZhy{}}\PYG{o}{\PYGZhy{}}\PYG{o}{\PYGZhy{}}\PYG{o}{\PYGZhy{}}\PYG{o}{\PYGZhy{}}\PYG{o}{\PYGZhy{}}\PYG{o}{\PYGZhy{}}\PYG{o}{\PYGZhy{}}\PYG{o}{\PYGZhy{}}\PYG{o}{\PYGZhy{}}\PYG{o}{\PYGZhy{}}\PYG{o}{\PYGZhy{}}\PYG{o}{\PYGZhy{}}\PYG{o}{\PYGZhy{}}\PYG{o}{\PYGZhy{}}\PYG{o}{\PYGZhy{}}
\PYG{n}{TraChan}\PYG{p}{:}\PYG{p}{:}\PYG{n}{mainRoutine}\PYG{p}{(}\PYG{p}{)}
\PYG{n}{site} \PYG{n}{occupancy} \PYG{n}{mode}
\PYG{n}{n\PYGZus{}atoms} \PYG{p}{:} \PYG{l+m+mi}{6997}
\PYG{n}{pore} \PYG{n}{axis} \PYG{n}{basis} \PYG{n}{a}
\PYG{l+m+mi}{1014} \PYG{n}{O} \PYG{n}{THR} \PYG{l+m+mi}{374}
\PYG{l+m+mi}{2735} \PYG{n}{O} \PYG{n}{THR} \PYG{l+m+mi}{374}
\PYG{l+m+mi}{4456} \PYG{n}{O} \PYG{n}{THR} \PYG{l+m+mi}{374}
\PYG{l+m+mi}{6177} \PYG{n}{O} \PYG{n}{THR} \PYG{l+m+mi}{374}
\PYG{n}{pore} \PYG{n}{axis} \PYG{n}{basis} \PYG{n}{b}
\PYG{l+m+mi}{1058} \PYG{n}{O} \PYG{n}{TYR} \PYG{l+m+mi}{377}
\PYG{l+m+mi}{2779} \PYG{n}{O} \PYG{n}{TYR} \PYG{l+m+mi}{377}
\PYG{l+m+mi}{4500} \PYG{n}{O} \PYG{n}{TYR} \PYG{l+m+mi}{377}
\PYG{l+m+mi}{6221} \PYG{n}{O} \PYG{n}{TYR} \PYG{l+m+mi}{377}
\PYG{n+nb}{open} \PYG{n}{pore\PYGZus{}axis}\PYG{o}{.}\PYG{n}{txt}
\PYG{n+nb}{open} \PYG{n}{pore\PYGZus{}axis\PYGZus{}r}\PYG{o}{.}\PYG{n}{txt}
\PYG{n}{average} \PYG{n}{axis} \PYG{n}{length} \PYG{o}{=} \PYG{l+m+mf}{0.945331}
\PYG{n}{sd} \PYG{n}{axis} \PYG{n}{length} \PYG{o}{=} \PYG{l+m+mf}{0.000977426}
\end{Verbatim}

And the two output files, \emph{pore\_axis.txt} and \emph{pore\_axis\_r.txt} will be obtained.

They are tab-separated text files. The first column indicates the time, and the other columns correspond to the coordinate of each potassium ion. Each row indicates the positions in each snapshot. \emph{pore\_axis.txt} and \emph{pore\_axis\_r.txt} record the coordinates along the pore axis and those along the radial direction parpendicular to the pore axis.


\section{Drawing a trajectory of ions along the pore axis}
\label{tutorial:drawing-a-trajectory-of-ions-along-the-pore-axis}
\begin{Verbatim}[commandchars=\\\{\}]
R \PYGZhy{}\PYGZhy{}vanilla \PYGZhy{}\PYGZhy{}slave \PYGZlt{} \PYGZdl{}IBISA/r/pore\PYGZus{}axis\PYGZus{}traj.R
\end{Verbatim}

\emph{pore\_axis\_traj.eps} shows the time course of ion coordinates.

\includegraphics[width=10cm]{{pore_axis_traj}.png}


\section{Analyzing a histogram of ions}
\label{tutorial:analyzing-a-histogram-of-ions}
Histogram of ion frequency over the 2D pore axis space can be drawn by:

\begin{Verbatim}[commandchars=\\\{\}]
python \PYGZdl{}IBISA/bin/ion\PYGZus{}histogram.py \PYGZbs{}
  \PYGZhy{}\PYGZhy{}i\PYGZhy{}pore\PYGZhy{}crd\PYGZhy{}h pore\PYGZus{}axis.txt \PYGZbs{}
  \PYGZhy{}\PYGZhy{}i\PYGZhy{}pore\PYGZhy{}crd\PYGZhy{}r pore\PYGZus{}axis\PYGZus{}r.txt \PYGZbs{}
  \PYGZhy{}\PYGZhy{}o\PYGZhy{}histogram histogram.txt \PYGZbs{}
  \PYGZhy{}\PYGZhy{}atomname K

R \PYGZhy{}\PYGZhy{}vanilla \PYGZhy{}\PYGZhy{}slave \PYGZlt{} \PYGZdl{}IBISA/r/histogram.R
\end{Verbatim}

\emph{histogram.eps} is the 1D and 2D histogram of ions.


\section{Analyzing the density distribution of ions along the pore axis}
\label{tutorial:analyzing-the-density-distribution-of-ions-along-the-pore-axis}
\begin{Verbatim}[commandchars=\\\{\}]
R \PYGZhy{}\PYGZhy{}vanilla \PYGZhy{}\PYGZhy{}slave \PYGZlt{} \PYGZdl{}IBISA/r/pore\PYGZus{}axis\PYGZus{}density.R
\end{Verbatim}

\emph{density\_distribution.eps} is the distribution plot.

\includegraphics[width=10cm]{{distribution}.png}

This plot clearly shows localization of ions in the ion-binding sites. On the basis of this plot, we can define the boundary of each ion binding site.


\section{Discretizing the trajectory based on ion-binding sites}
\label{tutorial:discretizing-the-trajectory-based-on-ion-binding-sites}
Here, we use the definition which is determined in our previous reports. The boundaries of ion binding sites are  15.13, 12.93, 9.32, 6.25, 3.00, 0.44, -2.21, -6.08, and -20.:

\begin{Verbatim}[commandchars=\\\{\}]
python \PYGZdl{}IBISA/bin/site\PYGZus{}occupancy.py \PYGZbs{}
 \PYGZhy{}\PYGZhy{}i\PYGZhy{}pore\PYGZhy{}crd\PYGZhy{}h pore\PYGZus{}axis.txt \PYGZbs{}
 \PYGZhy{}\PYGZhy{}i\PYGZhy{}pore\PYGZhy{}crd\PYGZhy{}r pore\PYGZus{}axis\PYGZus{}r.txt \PYGZbs{}
  \PYGZhy{}\PYGZhy{}o\PYGZhy{}site\PYGZhy{}occ   site\PYGZus{}occ.txt \PYGZbs{}
  \PYGZhy{}\PYGZhy{}atomname K \PYGZbs{}
  \PYGZhy{}b 12.93 \PYGZhy{}b 9.32 \PYGZhy{}b 6.25 \PYGZhy{}b 3.00 \PYGZhy{}b 0.44 \PYGZhy{}b \PYGZhy{}2.21 \PYGZhy{}b \PYGZhy{}6.08  \PYGZhy{}b \PYGZhy{}20 \PYGZbs{}
  \PYGZhy{}n \PYGZsq{}\PYGZhy{}1\PYGZsq{}  \PYGZhy{}n 0     \PYGZhy{}n 1    \PYGZhy{}n 2    \PYGZhy{}n 3    \PYGZhy{}n 4    \PYGZhy{}n 5     \PYGZhy{}n 6
\end{Verbatim}

The output file \emph{site\_occ.txt} records information about what ions are retained in each ion binding sites in each snapshot.

site\_ooc.txt:

\begin{Verbatim}[commandchars=\\\{\}]
\PYG{l+m+mi}{0}       \PYG{l+m+mi}{1}\PYG{p}{:}\PYG{l+m+mi}{6946}\PYG{p}{:}\PYG{n}{K}        \PYG{l+m+mi}{3}\PYG{p}{:}\PYG{l+m+mi}{6985}\PYG{p}{:}\PYG{n}{K}        \PYG{l+m+mi}{4}\PYG{p}{:}\PYG{l+m+mi}{6993}\PYG{p}{:}\PYG{n}{K}        \PYG{l+m+mi}{6}\PYG{p}{:}\PYG{l+m+mi}{6935}\PYG{p}{:}\PYG{n}{K}
\PYG{l+m+mi}{10}      \PYG{l+m+mi}{1}\PYG{p}{:}\PYG{l+m+mi}{6946}\PYG{p}{:}\PYG{n}{K}        \PYG{l+m+mi}{3}\PYG{p}{:}\PYG{l+m+mi}{6985}\PYG{p}{:}\PYG{n}{K}        \PYG{l+m+mi}{4}\PYG{p}{:}\PYG{l+m+mi}{6993}\PYG{p}{:}\PYG{n}{K}        \PYG{l+m+mi}{6}\PYG{p}{:}\PYG{l+m+mi}{6935}\PYG{p}{:}\PYG{n}{K}
\PYG{l+m+mi}{20}      \PYG{l+m+mi}{1}\PYG{p}{:}\PYG{l+m+mi}{6946}\PYG{p}{:}\PYG{n}{K}        \PYG{l+m+mi}{3}\PYG{p}{:}\PYG{l+m+mi}{6985}\PYG{p}{:}\PYG{n}{K}        \PYG{l+m+mi}{4}\PYG{p}{:}\PYG{l+m+mi}{6993}\PYG{p}{:}\PYG{n}{K}        \PYG{l+m+mi}{6}\PYG{p}{:}\PYG{l+m+mi}{6935}\PYG{p}{:}\PYG{n}{K}
\PYG{l+m+mi}{30}      \PYG{l+m+mi}{1}\PYG{p}{:}\PYG{l+m+mi}{6946}\PYG{p}{:}\PYG{n}{K}        \PYG{l+m+mi}{3}\PYG{p}{:}\PYG{l+m+mi}{6985}\PYG{p}{:}\PYG{n}{K}        \PYG{l+m+mi}{4}\PYG{p}{:}\PYG{l+m+mi}{6993}\PYG{p}{:}\PYG{n}{K}        \PYG{l+m+mi}{6}\PYG{p}{:}\PYG{l+m+mi}{6935}\PYG{p}{:}\PYG{n}{K}
\PYG{l+m+mi}{40}      \PYG{l+m+mi}{1}\PYG{p}{:}\PYG{l+m+mi}{6946}\PYG{p}{:}\PYG{n}{K}        \PYG{l+m+mi}{3}\PYG{p}{:}\PYG{l+m+mi}{6985}\PYG{p}{:}\PYG{n}{K}        \PYG{l+m+mi}{4}\PYG{p}{:}\PYG{l+m+mi}{6993}\PYG{p}{:}\PYG{n}{K}        \PYG{l+m+mi}{6}\PYG{p}{:}\PYG{l+m+mi}{6935}\PYG{p}{:}\PYG{n}{K}
\end{Verbatim}

``1:6956:K'' means the ion K with the ID 6946 is bound at the site 1.


\section{Analyzing the trajectories of each ion}
\label{tutorial:analyzing-the-trajectories-of-each-ion}
\begin{Verbatim}[commandchars=\\\{\}]
python \PYGZdl{}\PYGZob{}IBISA\PYGZcb{}/bin/analyze\PYGZus{}ion\PYGZus{}path.py \PYGZbs{}
  \PYGZhy{}\PYGZhy{}i\PYGZhy{}site\PYGZhy{}occ        site\PYGZus{}occ.txt \PYGZbs{}
  \PYGZhy{}\PYGZhy{}o\PYGZhy{}all\PYGZhy{}path        site\PYGZus{}path.txt \PYGZbs{}
  \PYGZhy{}\PYGZhy{}o\PYGZhy{}count\PYGZhy{}full      site\PYGZus{}path\PYGZus{}count\PYGZus{}full.txt \PYGZbs{}
  \PYGZhy{}\PYGZhy{}o\PYGZhy{}count\PYGZhy{}head\PYGZhy{}tail site\PYGZus{}path\PYGZus{}count\PYGZus{}ht.txt
\end{Verbatim}

The output \emph{site\_path.txt}:

\begin{Verbatim}[commandchars=\\\{\}]
\PYG{l+m+mi}{6985}    \PYG{n}{K}       \PYG{o}{*}\PYG{p}{:}\PYG{l+m+mi}{3}\PYG{p}{:}\PYG{l+m+mi}{0}\PYG{p}{:}\PYG{o}{*} \PYG{l+m+mi}{0}\PYG{p}{:}\PYG{l+m+mi}{7250}  \PYG{o}{*}\PYG{p}{:}\PYG{l+m+mi}{3}\PYG{p}{:}\PYG{l+m+mi}{2}\PYG{p}{:}\PYG{l+m+mi}{1}\PYG{p}{:}\PYG{l+m+mi}{0}\PYG{p}{:}\PYG{o}{*}     \PYG{l+m+mi}{0}\PYG{p}{:}\PYG{l+m+mi}{5290}\PYG{p}{:}\PYG{l+m+mi}{5340}\PYG{p}{:}\PYG{l+m+mi}{7240}\PYG{p}{:}\PYG{l+m+mi}{7250}
\PYG{l+m+mi}{6985}    \PYG{n}{K}       \PYG{o}{*}\PYG{p}{:}\PYG{l+m+mi}{0}\PYG{p}{:}\PYG{o}{*}   \PYG{l+m+mi}{7320}\PYG{p}{:}\PYG{l+m+mi}{7350}       \PYG{o}{*}\PYG{p}{:}\PYG{l+m+mi}{0}\PYG{p}{:}\PYG{o}{*}   \PYG{l+m+mi}{7320}\PYG{p}{:}\PYG{l+m+mi}{7350}
\PYG{l+m+mi}{6985}    \PYG{n}{K}       \PYG{o}{*}\PYG{p}{:}\PYG{l+m+mi}{0}\PYG{p}{:}\PYG{o}{*}   \PYG{l+m+mi}{7540}\PYG{p}{:}\PYG{l+m+mi}{7570}       \PYG{o}{*}\PYG{p}{:}\PYG{l+m+mi}{0}\PYG{p}{:}\PYG{o}{*}   \PYG{l+m+mi}{7540}\PYG{p}{:}\PYG{l+m+mi}{7570}
\end{Verbatim}
\begin{itemize}
\item {} 
The first column indicates the ID of the ion.

\item {} 
At the third column, ``\emph{:3:0:}'' means this ion got into the pore at site 3, and went out from the site 0.

\item {} 
The fourth column denote the times for getting into and going out from the pore.

\item {} 
The fifth column, ``\emph{:3:2:1:0:}'' indicates the full trajectory of this ion from association the to pore and dissociation from the pore.

\end{itemize}

Each line corresponds to each event starting with an ion association and ending with a dissociation of that ion. The third and fourth columns are abbreviations of the fifth and sixth columns, respectively.


\section{Generating the ion-binding state graph}
\label{tutorial:generating-the-ion-binding-state-graph}
\begin{Verbatim}[commandchars=\\\{\}]
python \PYGZdl{}IBISA/bin/analyze\PYGZus{}site\PYGZus{}state.py \PYGZbs{}
  \PYGZhy{}\PYGZhy{}i\PYGZhy{}site\PYGZhy{}occ site\PYGZus{}occ.txt \PYGZbs{}
  \PYGZhy{}\PYGZhy{}o\PYGZhy{}states   state\PYGZus{}traj.txt \PYGZbs{}
  \PYGZhy{}\PYGZhy{}o\PYGZhy{}graph    state\PYGZus{}graph.gml \PYGZbs{}
  \PYGZhy{}\PYGZhy{}atomname   K
\end{Verbatim}

The ion binding state graph can be visualized by using the output file \emph{state\_graph.gml} with a network analysis software, e.g., Cytoscape.

\includegraphics[width=10cm]{{network}.png}

\emph{state\_traj.txt} records the ion binding state in each snapshot:

\begin{Verbatim}[commandchars=\\\{\}]
\PYG{l+m+mi}{0}       \PYG{n}{K}\PYG{p}{:}\PYG{l+m+mi}{1}\PYG{p}{:}\PYG{l+m+mi}{3}\PYG{p}{:}\PYG{l+m+mi}{4}\PYG{p}{:}\PYG{l+m+mi}{6}       \PYG{n}{K}\PYG{p}{:}\PYG{l+m+mi}{6946}\PYG{p}{:}\PYG{l+m+mi}{6985}\PYG{p}{:}\PYG{l+m+mi}{6993}\PYG{p}{:}\PYG{l+m+mi}{6935}
\PYG{l+m+mi}{10}      \PYG{n}{K}\PYG{p}{:}\PYG{l+m+mi}{1}\PYG{p}{:}\PYG{l+m+mi}{3}\PYG{p}{:}\PYG{l+m+mi}{4}\PYG{p}{:}\PYG{l+m+mi}{6}       \PYG{n}{K}\PYG{p}{:}\PYG{l+m+mi}{6946}\PYG{p}{:}\PYG{l+m+mi}{6985}\PYG{p}{:}\PYG{l+m+mi}{6993}\PYG{p}{:}\PYG{l+m+mi}{6935}
\PYG{l+m+mi}{20}      \PYG{n}{K}\PYG{p}{:}\PYG{l+m+mi}{1}\PYG{p}{:}\PYG{l+m+mi}{3}\PYG{p}{:}\PYG{l+m+mi}{4}\PYG{p}{:}\PYG{l+m+mi}{6}       \PYG{n}{K}\PYG{p}{:}\PYG{l+m+mi}{6946}\PYG{p}{:}\PYG{l+m+mi}{6985}\PYG{p}{:}\PYG{l+m+mi}{6993}\PYG{p}{:}\PYG{l+m+mi}{6935}
\PYG{l+m+mi}{30}      \PYG{n}{K}\PYG{p}{:}\PYG{l+m+mi}{1}\PYG{p}{:}\PYG{l+m+mi}{3}\PYG{p}{:}\PYG{l+m+mi}{4}\PYG{p}{:}\PYG{l+m+mi}{6}       \PYG{n}{K}\PYG{p}{:}\PYG{l+m+mi}{6946}\PYG{p}{:}\PYG{l+m+mi}{6985}\PYG{p}{:}\PYG{l+m+mi}{6993}\PYG{p}{:}\PYG{l+m+mi}{6935}
\end{Verbatim}

The third column indicate the IDs of ions in the ion binding sites.


\section{Extracting cyclic paths from the state trajectory}
\label{tutorial:extracting-cyclic-paths-from-the-state-trajectory}
\begin{Verbatim}[commandchars=\\\{\}]
python \PYGZdl{}IBISA/bin/extract\PYGZus{}cycles.py \PYGZbs{}
  \PYGZhy{}\PYGZhy{}i\PYGZhy{}state  state\PYGZus{}traj.txt \PYGZbs{}
  \PYGZhy{}\PYGZhy{}o\PYGZhy{}cycles state\PYGZus{}traj\PYGZus{}cycles.txt \PYGZbs{}
  \PYGZhy{}\PYGZhy{}o\PYGZhy{}state\PYGZhy{}dict state\PYGZus{}dict\PYGZus{}pre.txt \PYGZbs{}
  \PYGZhy{}\PYGZhy{}title    \PYGZdq{}sample\PYGZdq{}
\end{Verbatim}
\begin{itemize}
\item {} 
option \emph{--title} is an arbitrary string.

\end{itemize}

\emph{state\_traj\_cycles.txt} stores the cyclic paths:

\begin{Verbatim}[commandchars=\\\{\}]
\PYG{o}{\PYGZgt{}}       \PYG{l+m+mi}{1}       \PYG{l+m+mi}{6010}    \PYG{l+m+mi}{7830}    \PYG{n}{sample}
\PYG{l+m+mi}{6010}    \PYG{n}{K}\PYG{p}{:}\PYG{l+m+mi}{0}\PYG{p}{:}\PYG{l+m+mi}{2}\PYG{p}{:}\PYG{l+m+mi}{4} \PYG{n}{K}\PYG{p}{:}\PYG{l+m+mi}{6985}\PYG{p}{:}\PYG{l+m+mi}{6993}\PYG{p}{:}\PYG{l+m+mi}{6935}
\PYG{l+m+mi}{6630}    \PYG{n}{K}\PYG{p}{:}\PYG{l+m+mi}{0}\PYG{p}{:}\PYG{l+m+mi}{2}\PYG{p}{:}\PYG{l+m+mi}{4}\PYG{p}{:}\PYG{l+m+mi}{6}       \PYG{n}{K}\PYG{p}{:}\PYG{l+m+mi}{6985}\PYG{p}{:}\PYG{l+m+mi}{6993}\PYG{p}{:}\PYG{l+m+mi}{6935}\PYG{p}{:}\PYG{l+m+mi}{6961}
\PYG{l+m+mi}{7190}    \PYG{n}{K}\PYG{p}{:}\PYG{l+m+mi}{0}\PYG{p}{:}\PYG{l+m+mi}{2}\PYG{p}{:}\PYG{l+m+mi}{4}\PYG{p}{:}\PYG{l+m+mi}{5}       \PYG{n}{K}\PYG{p}{:}\PYG{l+m+mi}{6985}\PYG{p}{:}\PYG{l+m+mi}{6993}\PYG{p}{:}\PYG{l+m+mi}{6935}\PYG{p}{:}\PYG{l+m+mi}{6961}
\PYG{l+m+mi}{7230}    \PYG{n}{K}\PYG{p}{:}\PYG{l+m+mi}{0}\PYG{p}{:}\PYG{l+m+mi}{1}\PYG{p}{:}\PYG{l+m+mi}{3}\PYG{p}{:}\PYG{l+m+mi}{5}       \PYG{n}{K}\PYG{p}{:}\PYG{l+m+mi}{6985}\PYG{p}{:}\PYG{l+m+mi}{6993}\PYG{p}{:}\PYG{l+m+mi}{6935}\PYG{p}{:}\PYG{l+m+mi}{6961}
\PYG{l+m+mi}{7820}    \PYG{n}{K}\PYG{p}{:}\PYG{l+m+mi}{1}\PYG{p}{:}\PYG{l+m+mi}{3}\PYG{p}{:}\PYG{l+m+mi}{5} \PYG{n}{K}\PYG{p}{:}\PYG{l+m+mi}{6993}\PYG{p}{:}\PYG{l+m+mi}{6935}\PYG{p}{:}\PYG{l+m+mi}{6961}
\PYG{l+m+mi}{7830}    \PYG{n}{K}\PYG{p}{:}\PYG{l+m+mi}{0}\PYG{p}{:}\PYG{l+m+mi}{2}\PYG{p}{:}\PYG{l+m+mi}{4} \PYG{n}{K}\PYG{p}{:}\PYG{l+m+mi}{6993}\PYG{p}{:}\PYG{l+m+mi}{6935}\PYG{p}{:}\PYG{l+m+mi}{6961}
\end{Verbatim}
\begin{itemize}
\item {} 
The line begining with ``\textgreater{}'' is the header line. The cyclic path ``1'' starts at 6010 and ends at 7830.

\end{itemize}


\section{Converting states into characters. A cyclic parts transformed into a sequence::}
\label{tutorial:converting-states-into-characters-a-cyclic-parts-transformed-into-a-sequence}
\begin{Verbatim}[commandchars=\\\{\}]
python \PYGZdl{}IBISA/bin/cycle\PYGZus{}to\PYGZus{}sequence.py \PYGZbs{}
  \PYGZhy{}\PYGZhy{}i\PYGZhy{}cycles     state\PYGZus{}traj\PYGZus{}cycles.txt \PYGZbs{}
  \PYGZhy{}\PYGZhy{}i\PYGZhy{}state\PYGZhy{}dict state\PYGZus{}dict\PYGZus{}pre.txt \PYGZbs{}
  \PYGZhy{}\PYGZhy{}o\PYGZhy{}state\PYGZhy{}dict state\PYGZus{}dict.txt \PYGZbs{}
  \PYGZhy{}\PYGZhy{}o\PYGZhy{}sequence   sequences.fsa
\end{Verbatim}
\begin{itemize}
\item {} 
\emph{state\_dict.txt} describes a correspondence between states and characters.

\item {} 
\emph{sequences.fsa} is the sequences of cyclic paths.

\end{itemize}

\begin{Verbatim}[commandchars=\\\{\}]
\PYG{o}{\PYGZgt{}} \PYG{l+m+mi}{0}     \PYG{l+m+mi}{0}       \PYG{o}{*}\PYG{n}{POMFB}\PYG{o}{*} \PYG{l+m+mi}{5}       \PYG{l+m+mi}{28990}   \PYG{l+m+mi}{31650}   \PYG{n}{sample}
\PYG{o}{*}\PYG{n}{POMFB}\PYG{o}{*}
\PYG{o}{\PYGZgt{}} \PYG{l+m+mi}{1}     \PYG{l+m+mi}{0}       \PYG{o}{*}\PYG{n}{POMFE}\PYG{o}{*} \PYG{l+m+mi}{8}       \PYG{l+m+mi}{44310}   \PYG{l+m+mi}{45210}   \PYG{n}{sample}
\PYG{o}{*}\PYG{n}{POMFE}\PYG{o}{*}
\PYG{o}{\PYGZgt{}} \PYG{l+m+mi}{2}     \PYG{l+m+mi}{0}       \PYG{o}{*}\PYG{n}{PJHF}\PYG{o}{*}  \PYG{l+m+mi}{7}       \PYG{l+m+mi}{39280}   \PYG{l+m+mi}{42200}   \PYG{n}{sample}
\PYG{o}{*}\PYG{n}{PJHF}\PYG{o}{*}
\PYG{o}{\PYGZgt{}} \PYG{l+m+mi}{3}     \PYG{l+m+mi}{0}       \PYG{o}{*}\PYG{n}{POKLF}\PYG{o}{*} \PYG{l+m+mi}{6}       \PYG{l+m+mi}{36850}   \PYG{l+m+mi}{38640}   \PYG{n}{sample}
\PYG{o}{*}\PYG{n}{POKLF}\PYG{o}{*}
\end{Verbatim}

Assignments of characters for states can be modified by revising the file specified as --i-state-dict option. For clarity, the characters for the ion binding states with 4 ions are replaced with lower cases, and 5-ion state is ``\#''.

state\_dict.txt:

\begin{Verbatim}[commandchars=\\\{\}]
\PYG{n}{K}\PYG{p}{:}\PYG{l+m+mi}{2}\PYG{p}{:}\PYG{l+m+mi}{3}\PYG{p}{:}\PYG{l+m+mi}{6} \PYG{n}{I}
\PYG{n}{K}\PYG{p}{:}\PYG{l+m+mi}{2}\PYG{p}{:}\PYG{l+m+mi}{4}\PYG{p}{:}\PYG{l+m+mi}{6} \PYG{n}{J}
\PYG{n}{K}\PYG{p}{:}\PYG{l+m+mi}{0}\PYG{p}{:}\PYG{l+m+mi}{1}\PYG{p}{:}\PYG{l+m+mi}{3}\PYG{p}{:}\PYG{l+m+mi}{5}       \PYG{n}{k}
\PYG{n}{K}\PYG{p}{:}\PYG{l+m+mi}{0}\PYG{p}{:}\PYG{l+m+mi}{1}\PYG{p}{:}\PYG{l+m+mi}{3}\PYG{p}{:}\PYG{l+m+mi}{6}       \PYG{n}{l}
\PYG{n}{K}\PYG{p}{:}\PYG{l+m+mi}{0}\PYG{p}{:}\PYG{l+m+mi}{2}\PYG{p}{:}\PYG{l+m+mi}{3}\PYG{p}{:}\PYG{l+m+mi}{5}       \PYG{n}{m}
\PYG{o}{.}\PYG{o}{.}
\PYG{n}{K}\PYG{p}{:}\PYG{l+m+mi}{0}\PYG{p}{:}\PYG{l+m+mi}{2}\PYG{p}{:}\PYG{l+m+mi}{3}\PYG{p}{:}\PYG{l+m+mi}{5}\PYG{p}{:}\PYG{l+m+mi}{6}     \PYG{c+c1}{\PYGZsh{}}
\end{Verbatim}

Then, re-do \emph{cycle\_to\_sequence.py}:

\begin{Verbatim}[commandchars=\\\{\}]
python \PYGZdl{}IBISA/bin/cycle\PYGZus{}to\PYGZus{}sequence.py \PYGZbs{}
  \PYGZhy{}\PYGZhy{}i\PYGZhy{}cycles     state\PYGZus{}traj\PYGZus{}cycles.txt \PYGZbs{}
  \PYGZhy{}\PYGZhy{}i\PYGZhy{}state\PYGZhy{}dict state\PYGZus{}dict.txt \PYGZbs{}
  \PYGZhy{}\PYGZhy{}o\PYGZhy{}sequence   sequences.fsa
\end{Verbatim}


\section{Generating score matrix of states}
\label{tutorial:generating-score-matrix-of-states}
\begin{Verbatim}[commandchars=\\\{\}]
python \PYGZdl{}IBISA/bin/make\PYGZus{}score\PYGZus{}matrix.py \PYGZbs{}
   \PYGZhy{}\PYGZhy{}i\PYGZhy{}state\PYGZhy{}dict  state\PYGZus{}dict.txt \PYGZbs{}
   \PYGZhy{}\PYGZhy{}o\PYGZhy{}score       score\PYGZus{}matrix.txt
\end{Verbatim}

For each pair of ion-binding states, when the two states are identical, the similarity score is 1.0. When the two states have the same number of ions, the score is 0.5. Otherwise, teh score is 0.0.


\section{Performing the sequence alignment}
\label{tutorial:performing-the-sequence-alignment}
\begin{Verbatim}[commandchars=\\\{\}]
python \PYGZdl{}IBISA/bin/dp\PYGZus{}align.py \PYGZbs{}
   \PYGZhy{}\PYGZhy{}i\PYGZhy{}score\PYGZhy{}matrix score\PYGZus{}matrix.txt \PYGZbs{}
   \PYGZhy{}\PYGZhy{}i\PYGZhy{}sequence     sequences.fsa \PYGZbs{}
   \PYGZhy{}\PYGZhy{}o\PYGZhy{}align        align.txt \PYGZhy{}a\PYGZbs{}
   \PYGZhy{}\PYGZhy{}min\PYGZhy{}len   4 \PYGZbs{}
   \PYGZhy{}g 1.0 \PYGZbs{}
   \PYGZhy{}m 1.0 \PYGZbs{}
   \PYGZhy{}\PYGZhy{}ignore *
\end{Verbatim}
\begin{itemize}
\item {} 
\emph{-g} and \emph{-m} are gap score and match scores, respectively.

\item {} 
The output file \emph{align.txt} shows the pairwise alignments

\end{itemize}

\begin{Verbatim}[commandchars=\\\{\}]
\PYG{o}{\PYGZgt{}} \PYG{l+m+mi}{2}     \PYG{l+m+mi}{3}       \PYG{l+m+mf}{0.0}     \PYG{l+m+mi}{2}       \PYG{l+m+mi}{0}       \PYG{o}{*}\PYG{n}{pJHF}\PYG{o}{*}  \PYG{l+m+mi}{7}       \PYG{o}{.}\PYG{o}{.}\PYG{o}{.}
\PYG{o}{*}\PYG{n}{pJH}\PYG{o}{\PYGZhy{}}\PYG{n}{F}\PYG{o}{*}
\PYG{o}{*}\PYG{n}{poklF}\PYG{o}{*}
\PYG{o}{\PYGZgt{}} \PYG{l+m+mi}{0}     \PYG{l+m+mi}{3}       \PYG{l+m+mf}{1.0}     \PYG{l+m+mi}{0}       \PYG{l+m+mi}{0}       \PYG{o}{*}\PYG{n}{pomFB}\PYG{o}{*} \PYG{l+m+mi}{5}       \PYG{o}{.}\PYG{o}{.}\PYG{o}{.}
\PYG{o}{*}\PYG{n}{pom}\PYG{o}{\PYGZhy{}}\PYG{n}{FB}\PYG{o}{*}
\PYG{o}{*}\PYG{n}{poklF}\PYG{o}{\PYGZhy{}}\PYG{o}{*}
\end{Verbatim}


\section{Make the similarity matrix of cyclic paths}
\label{tutorial:make-the-similarity-matrix-of-cyclic-paths}
\begin{Verbatim}[commandchars=\\\{\}]
python \PYGZdl{}IBISA/bin/align\PYGZus{}similarity.py \PYGZbs{}
  \PYGZhy{}\PYGZhy{}i\PYGZhy{}align     align.txt \PYGZbs{}
  \PYGZhy{}\PYGZhy{}i\PYGZhy{}sequence  sequences.fsa \PYGZbs{}
  \PYGZhy{}\PYGZhy{}o\PYGZhy{}sim       align\PYGZus{}sim.txt \PYGZbs{}
  \PYGZhy{}g 1 \PYGZhy{}m 1
\end{Verbatim}


\section{Clustering aligned sequences by using R}
\label{tutorial:clustering-aligned-sequences-by-using-r}
\begin{Verbatim}[commandchars=\\\{\}]
R \PYGZhy{}\PYGZhy{}vanilla \PYGZhy{}\PYGZhy{}slave \PYGZlt{} \PYGZdl{}IBISA/r/clustering\PYGZus{}seq.R
\end{Verbatim}

\includegraphics[width=10cm]{{dendrogram}.png}


\chapter{Appendix}
\label{appendix:appendix}\label{appendix::doc}

\section{Converting a trajectory file}
\label{appendix:converting-a-trajectory-file}
\emph{IBiSA\_tools} can read only .trr format. When you analyze trajectories written in other format, the files have to be converted into .trr file.

Some tools for the conversion exist. Here, a sample code powered by the Python library \emph{MDAnalysis} is presented.:

\begin{Verbatim}[commandchars=\\\{\}]
\PYG{k+kn}{import} \PYG{n+nn}{MDAnalysis}

\PYG{n}{u} \PYG{o}{=} \PYG{n}{MDAnalysis}\PYG{o}{.}\PYG{n}{Universe}\PYG{p}{(}\PYG{l+s+s2}{\PYGZdq{}}\PYG{l+s+s2}{initial.pdb}\PYG{l+s+s2}{\PYGZdq{}}\PYG{p}{,}\PYG{l+s+s2}{\PYGZdq{}}\PYG{l+s+s2}{trajectory.ncdf}\PYG{l+s+s2}{\PYGZdq{}}\PYG{p}{)}
\PYG{n}{writer} \PYG{o}{=} \PYG{n}{MDAnalysis}\PYG{o}{.}\PYG{n}{coordinates}\PYG{o}{.}\PYG{n}{TRJ}\PYG{o}{.}\PYG{n}{TRRWriter}\PYG{p}{(}\PYG{l+s+s2}{\PYGZdq{}}\PYG{l+s+s2}{trajectory.trr}\PYG{l+s+s2}{\PYGZdq{}}\PYG{p}{,} \PYG{n+nb}{len}\PYG{p}{(}\PYG{n}{u}\PYG{o}{.}\PYG{n}{atoms}\PYG{p}{)}\PYG{p}{)}
\PYG{k}{for} \PYG{n}{ts} \PYG{o+ow}{in} \PYG{n}{u}\PYG{o}{.}\PYG{n}{trajectory}\PYG{p}{:}
    \PYG{n}{writer}\PYG{o}{.}\PYG{n}{write\PYGZus{}next\PYGZus{}timestep}\PYG{p}{(}\PYG{n}{ts}\PYG{p}{)}
\end{Verbatim}

The script converts \emph{trajectory.ncdf} (AMBER, NetCDF file) into \emph{trajectory.trr} file (GROMACS, .trr file). \emph{initial.pdb} is the initial coordinates writtein in the flat .pdb format.



\renewcommand{\indexname}{Index}
\printindex
\end{document}
